%!TEX TS-program = xelatex
%!TEX encoding = UTF-8 Unicode
%!TEX root = 2022-GS-ARTICLE.tex
%----------------------------------------------------------------- LANGUAGES ---
\newcommand{\mylanguages}{italian} % in reverse order
%---------------------------------------------------------- TITLE & SUBTITLE ---
\newcommand{\mytitle}{Totalità sonora: il corpo dell'orchestra}
\newcommand{\mysubtitle}{Per ascoltare l'organismo sonoro orchestrale nel suo complesso e nei singoli dettagli\\ da lontano e da vicino}
%----------------------------------------------------------------- AUTHOR(s) ---
\newcommand{\authorone}{Giancarlo Bottalico}
\newcommand{\institutione}{Conservatorio di musica "N. Piccinni", Bari}
\newcommand{\emailone}{giancarlobottalico@gmail.com}
%-------------------------------------------------------------------------------
% \newcommand{\authortwo}{Wikio Orgopedio}
% \newcommand{\institutiontwo}{Conservatorio S. Cecilia di Roma}
% \newcommand{\emailtwo}{wikio @ orgopedio.com} % duplicate these 3 lines if more
%-------------------------------------------------------------- STYLE GS2020 ---
\input{gs2022.tex}
%------------------------------------------------------------ BEGIN DOCUMENT ---
\begin{document}
\maketitle
\thispagestyle{empty}
%-------------------------------------------------------------------- ABSTRACT -
% The abstract is an external txt file inside the includes folder
%-------------------------------------------------------------------------------
\section*{Disposizione dei microfoni}
Per la ripresa dell'intera orchestra è stata utilizzata una coppia stereo ORTF oltre a  10 microfoni omni direzionali di modello Omni1 della Line Audio organizzati in coppie stereofoniche, disposti nello spazio come descritto nella mappa della figura 2.

\subsection*{Ruoli di ciascun punto di ripresa}
Ciascuno dei microfoni ha un ruolo preciso all'interno della ripresa dell'orchestra, in base a due parametri principali, ovvero:
A Coppia stereofonica della quale è parte
B Il suo posizionamento nello spazio: essendo i microfoni utilizzati identici tra loro (fatta eccezione per l'ORTF), il loro carattere dipende solo ed esclusivamente dal loro posizionamento. E' solo in base a questo che la sonorità dell'ambiente e della fonte sonora cambia nella ripresa.

\subsubsection{L'ORTF}
Il punto di ripresa principale è costituito da un ORTF, ovvero una coppia stereofonica semi spaziata con una distanza di 17cm circa tra le due capsule (circa la stessa distanza che intercorre tra le orecchie in un cranio umano) inclinate di 110° circa che permette di riprendere l'ambiente in maniera tridimensionale, fornendo informazioni relative alla profondità dell'ambiente acustico.
La caratteristica principale del modello ORTF sta nella sua naturalezza nel catturare l'ambiente: diversamente da una situazione in ambiente DAW, questa coppia semi-spaziata permette di catturare le fonti con i loro rispettivi ritardi, fornendo informazioni sulla reale provenienza del suono in base al ritardo che intercorre tra le due capsule microfoniche nel catturarla.
In questo caso, L'ORTF è stato posto di fronte all'orchestra in corrispondenza del pianoforte e del direttore d'orchestra, a fronte palco.

\subsubsection {I 10 Line Audio modello Omni1}
Anzitutto, a proposito dei microfoni omni-direzionali, c'è da precisare che questo tipo di microfono non ha una linearità nella direzionalità: si comportano diversamente a seconda delle frequenze che catturano. Certo non sono tutti uguali dunque questo andamento non lineare è più accentuato in alcuni microfoni e meno in altri, ma in linee generiche la tendenza è quella di limitare la sua direzionalità al crescere delle frequenze catturate stringendo la figura polare rendendola più simile a quella di un cardioide. Nella figura 1 è presente il diagramma polare e la risposta in frequenza del Line Audio - Omni1.
I 10 microfoni sono stati così disposti:

Uno per i primi violini ed uno per le viole, posizionati specularmente nell'orchestra a distanza di 425cm l'uno dll'altro.
Tre per i fiati, dei quali uno centrale e due laterali, per allargarne l'immagine. Per questa sezione sono stati utilizzati tre microfoni per due motivi principali:
A La sezione è molto ampia(428cm)
B Per catturare anche i timpani, e in particolar modo il loro dettaglio, essendo storicamente utilizzati per dare sostegno ai fiati aggiungendo attacco e corpo al  suono.

Due in configurazione stereo AB L/R a distanza di 28cm l'uno dall'altro per il pianoforte per catturarne il dettaglio da vicino.

Uno come spot per i contrabbassi, posto a 309cm dal flankR, solo per poter compensare il loro dettaglio che sicuramente manca al resto: la sua utilità si limita a questo, in quanto non potrebbe dare informazioni sulla profondità della sezione trattandosi di un microfono mono, così come non potrebbe riprenderne fedelmente le basse frequenze essendo un microfono di spot e quindi molto vicino alla sorgente, anche perchè le basse frequenze dei contrabbassi, così come degli altri strumenti (in particolar modo dei timpani), saranno già riprese fedelmente dai microfoni main, essendo questi molto distanti dalla sorgente.

Due utilizzati come Flanks, ovvero di sostegno all'ORTF a distanza speculare di 324cm per allargarne l'immagine. Il complesso Flanks + ORTF costituisce la ripresa MAIN dell'orchestra, ovvero quella generale, la quale non serve a catturare i singoli dettagli, quanto a dare un'immagine complessiva frontale del corpo dell'orchestra.


\section*{Gestione del materiale}

\subsection*{Sessione di registrazione}
Come interfaccia audio è stato utilizzato il mixer digitale di marca Allen e Heat, modello SQ5. I canali sono stati mappati sulla DAW Reaper. Dopo aver impostato i volumi allo stesso livello e acceso le Phantom power su tutti i canali, è stata avviata la registrazione. 

\subsection*{Trattamento del materiale Mid/Side}
Perchè trattare il materiale delle registrazioni in Mid Side? In questa configurazione, è possibile gestire l'ambiente come uno spazio trigonometrico, in quanto la gestione dell'ambiente acustico non è solo relativa alle ampiezze delle riflessioni come in configurazione Left/Right tradizionale(la quale non fornisce realmente unn'immagine stereofonica ma solo una sua approssimazione), ma è possibile percepire la profondità di una fonte sonora nello spazio.
Ma procediamo per steps.
Tutti gli spot utilizzati in fase di registrazione sono mono, dunque il primo step è sicuramente metterli in correlazione con il punto di ripresa MAIN per poterci fare un'idea del posizionamento che  gli vogliamo dare nell'immagine stereo.


\vfill\null

\begin{figure}[b]
\begin{center}
\includegraphics[width=.47\textwidth]{img/image1.png}
\caption{\textbf{Caratteristiche del microfono di marca Line Audio modello Omni1}. Diagramma polare e risposta in frequenza
of spacetime caused by a planetary mass.}
\label{gr01}
\end{center}
\end{figure}

\newpage % USE NEWPAGE TO FORCE COLUMNN INTERRUPTION
%-------------------------------------------------------------------------------
%-------------------------------------------------------------------------------

\begin{table}[htp]
\begin{tabular}{ll}
\textbf{Sommario} & \textbf{Page} \\
\hline
\textbf{Disposizione dei microfoni} & 1 \\
Ruoli di ciascun punto di ripresa & \\
L'ORTF & \\
I 10 Line Audio modello Omni1 & \\
\hline
\textbf{Gestione del materiale} & 2 \\
Sessione di registrazione & \\
Trattamento del materiale Mid/Side & \\

\end{tabular}
\end{table}

\begin{figure}[t]
\centering
\includegraphics[width=.47\textwidth]{img/image2.jpg}
\caption{Mappa della disposizione dei microfoni}
\label{gs}
\end{figure}

\begin{equation}
Mid = Left(-3dB) + Right(-3dB)\\Side = Left(-3dB) + Right(-3dB)
\label{eq:mid}
\end{equation}

%--------------------------------------------
%----------------larghezza massima del codice

\vfill\null

\raggedright
%\bibliographystyle{unsrt}
%\printbibliography

\end{document}

%%%%%%%%%%%%%%%%%%%%%%%%%%%%%%%%%%%%%%%%%%%%%%%%%%%%%%%%%%%%%%%%%%%%%%%%%%%%%%%%
% 2020 GIUSEPPE SILVI ARTICLE TEMPLATE BASED ON
%%%%%%%%%%%%%%%%%%%%%%%%%%%%%%%%%%%%%%%%%%%%%%%%%%%%%%%%%%%%%%%%%%%%%%%%%%%%%%%%
% Journal Article
% LaTeX Template
% Version 1.4 (15/5/16)
% This template has been downloaded from:
% http://www.LaTeXTemplates.com
% Original author:
% Frits Wenneker (http://www.howtotex.com) with extensive modifications by
% Vel (vel@LaTeXTemplates.com)
% License:
% CC BY-NC-SA 3.0 (http://creativecommons.org/licenses/by-nc-sa/3.0/)
%%%%%%%%%%%%%%%%%%%%%%%%%%%%%%%%%%%%%%%%%%%%%%%%%%%%%%%%%%%%%%%%%%%%%%%%%%%%%%%%
